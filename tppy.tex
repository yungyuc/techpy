%%
%% Copyright (c) 2016 Yung-Yu Chen <yyc@solvcon.net>
%%
%% All rights reserved.
%%

\documentclass[a4paper,12pt]{book}
%\documentclass[a4paper,12pt]{article}
%\documentclass[a4paper,12pt,dvips]{article}

\usepackage[textwidth=6.5in,textheight=9in]{geometry}
\usepackage[colorlinks=true]{hyperref}
\usepackage[monochrome]{color}

\usepackage{fontspec}
\usepackage{xeCJK}

\setmainfont{Calibri}
\setCJKmainfont{Noto Sans CJK TC}
%\setCJKmainfont{Microsoft JhengHei}
%\setCJKmainfont{LiHei Pro}
%\setromanfont{LiHei Pro} % 儷黑Pro
%\setmonofont{Courier New} % 等寬字型
\XeTeXlinebreaklocale "zh"
\XeTeXlinebreakskip = 0pt plus 1pt

\usepackage[nopostdot]{glossaries}

\usepackage{amsmath}
\usepackage{amssymb}

\usepackage{amsthm}
\theoremstyle{definition}
\newtheorem{example}{範例}[chapter]

\usepackage{graphicx}     % From LaTeX distribution
\graphicspath{{eps/}}

\usepackage[center,footnotesize]{caption}
\usepackage[subrefformat=parens]{subcaption}
%\usepackage{subfigure}    % From CTAN/macros/latex/contrib/supported/subfigure
\usepackage{pst-all}      % From PSTricks
\usepackage{pst-poly}     % From pstricks/contrib/pst-poly
\usepackage{multido}      % From PSTricks

\usepackage{xCJKnumb}
\usepackage{titlesec}
\usepackage{titletoc}
%\titleformat{\part}{\centering\Huge\bfseries}{第\,\thepart\,部分}{1em} {}
%\titleformat{\chapter}{\centering\huge\bfseries}{第\,\thechapter\,章}{1em} {}
%\titleformat{\chapter}{\centering\Huge\bfseries}{第\,\xCJKnumber{\thechapter}\,章}{1em} {}

\titleclass{\part}{top}
\titleformat{\part}%[display]
  {\normalfont\huge\bfseries}
  {第\xCJKnumber{\arabic{part}}部分}
  {1em} {}
\titlespacing*{\part}{0pt}{50pt}{40pt}

\titleclass{\chapter}{straight}
\titleformat{\chapter}%[display]
  {\normalfont\huge\bfseries}
  {第\xCJKnumber{\arabic{chapter}}章}
  {1em} {}
\titlespacing*{\chapter} {0pt}{50pt}{40pt}

\usepackage{parskip}

\usepackage{etoolbox}
\makeatletter
%\patchcmd{\chapter}{}{}{}{}
\patchcmd{\chapter}{\if@openright\cleardoublepage\else\clearpage\fi}{}{}{}
\makeatother

\renewcommand{\contentsname}{目錄}
\renewcommand{\listfigurename}{圖目錄}
\newcommand{\LOFname}{圖目錄}
\newcommand{\LOTname}{表目錄}
\renewcommand{\listtablename}{表目錄}
\renewcommand{\chaptername}{章}
\renewcommand{\appendixname}{附錄}
\newcommand{\appendices}{附錄}
\newcommand{\pagename}{頁}
\renewcommand{\glossaryname}{詞彙表}

\begin{document}

\title{Python 計算實務}
\author{Yung-Yu Chen}
\date{2016.12.6}

\maketitle

\tableofcontents
%\listoffigures

\hspace{.5cm}

\frontmatter

\chapter*{前言}
\addcontentsline{toc}{chapter}{前言}

本書是為了沒有學習過任何程式語言,尤其是非資訊科系的大學部同學,
所提供的 Python 入門教材。

Python 是一個簡潔、自然的語言,長期得到程式員的喜愛。
近十年來,Python 都是 TIOBE 社群指標\cite{tiobe_index}中的前十名,
並兩度入選年度程式語言 ("Programming Language of the Year")。
IEEE 最近公布了一項調查\cite{ieee_pl_2016},
其中 Python 是第三受歡迎的程式語言,
僅次於 C 和 Java,並且是最受歡迎的動態語言 (dynamic language)。

程式語言為了驅使電腦而存在。
但實際在開發軟體時,閱讀程式碼的時間遠多於寫出程式碼的時間。
考慮到這一點,Python 被設計成容易閱讀的程式語言。
用 C 或是 C++ 要花一個小時寫的程式,使用 Python,常常只需要五分鐘。

Python 的可讀性使 Python 程式容易被維護。
程式寫出來以後仍然需要時常維護修改。
維護時不會增加新功能,但要閱讀程式,並修改那些一開始寫得不夠完美的部分。
沒有人能一次就把程式寫到盡善盡美。如同作家無時無刻都在修改文稿一樣。
如果程式寫得不容易讀懂,時間一久,連原本的開發者也會難以修改。

因為 Python 易學、易維護,許多複雜的計算系統都選擇使用 Python,
為進階使用者提供應用程式介面 (API)。
這些計算系統,為了執行效率,往往選擇使用 C++ 開發。
C++ 的執行效率雖然極好,但是非常複雜難學,很難用來提供高階的 API。
所以許多這些系統會提供 Python 介面,讓使用者寫程式操作。
因為這樣的應用很常見,也有許多程式庫專門協助 C++ 程式建立 Python 介面。

\mainmatter

\chapter{安裝執行環境}

本書使用 Ubuntu Linux 作為執行平台。
使用 Apple OSX 的讀者可以用類似的方式執行本書中的範例程式。
使用 Windows 的讀者需要在虛擬機器中安裝 Ubuntu Linux。

本書針對 Python 3,而非 Python 2。

\section{Ubuntu}

\section{VirtualBox}

\section{Anaconda}

\begin{example}
使用命令列輸出 Hello World
\end{example}

\chapter{命令列環境}

\chapter{編輯器}

\chapter{數值和字串、表達式和陳述式}

\section{整數與布林運算}

布林代數與位元運算。

\begin{example}
位元平移
\end{example}

\begin{example}
實作真值表
\end{example}

\section{實數運算}

四則運算。包含整數與實數的混合運算。

\begin{example}
實作雙元運算子計算機
\end{example}

\begin{example}
真實除法與整數除法不一樣
\end{example}

\section{字串}

Python 使用萬國碼 (Unicode) 字串 (string)。
但萬國碼字串並不是最基礎的字串表示方式。
為了與低階的程式相容 (例如 C),Python 另外提供一種位元組 (bytes) 容器。

\begin{example}
計算字串的長度
\end{example}

\begin{example}
計算字元出現的次數
\end{example}

\chapter{資料結構、演算法、容器}

\section{序列}

\section{數組}

\section{列表}

\section{字典}

\section{集合}

\chapter{迴圈與條件判斷}

縮排。

\chapter{函式、類別、物件}

\section{遞迴}

\chapter{例外處理}

\chapter{迭代器與產生器}

\chapter{檔案與輸出入}

\chapter{測試與除錯}

\chapter{佈署、模組、套件}

匯入。

相對匯入。

避免全域變數。

\chapter{詮釋編程}

\chapter{並發計算}

\chapter{進階使用}

\begin{itemize}

\item 載入網路資料;JSON。

\item NumPy 多維連續陣列。

\item 線性系統。

\item 影像處理。

\item 訊號分析。

\end{itemize}

\backmatter

\addcontentsline{toc}{chapter}{參考資料}
\bibliographystyle{unsrt}
\bibliography{bibliography}

\end{document}

% vim: set nobomb ff=unix fenc=utf8:
