%%
%% Copyright (c) 2016 Yung-Yu Chen <yyc@solvcon.net>
%%
%% All rights reserved.
%%

\documentclass[a4paper,12pt]{book}
%\documentclass[a4paper,12pt]{article}
%\documentclass[a4paper,12pt,dvips]{article}

\usepackage[textwidth=6.5in,textheight=9in]{geometry}

\usepackage{fontspec}
\usepackage{xeCJK}
\setCJKmainfont{LiHei Pro}
%\setromanfont{LiHei Pro} % 儷黑Pro
%\setmonofont{Courier New} % 等寬字型
\XeTeXlinebreaklocale "zh"
\XeTeXlinebreakskip = 0pt plus 1pt

\usepackage[nopostdot]{glossaries}

\usepackage[colorlinks=true]{hyperref}
\usepackage{amsmath}
\usepackage{amssymb}
\usepackage{amsthm}
\usepackage[monochrome]{color}
\usepackage{graphicx}     % From LaTeX distribution
%\usepackage{subfigure}    % From CTAN/macros/latex/contrib/supported/subfigure
\usepackage{pst-all}      % From PSTricks
\usepackage{pst-poly}     % From pstricks/contrib/pst-poly
\usepackage{multido}      % From PSTricks
\usepackage[center,footnotesize]{caption}
\usepackage[subrefformat=parens]{subcaption}

\usepackage{titlesec}
\usepackage{titletoc}
\usepackage{xCJKnumb}
\titleformat{\chapter}{\centering\Huge\bfseries}{第\,\thechapter\,章}{1em} {}
%\titleformat{\chapter}{\centering\Huge\bfseries}{第\,\xCJKnumber{\thechapter}\,章}{1em} {}

\usepackage{etoolbox}
\makeatletter
%\patchcmd{\chapter}{}{}{}{}
\patchcmd{\chapter}{\if@openright\cleardoublepage\else\clearpage\fi}{}{}{}
\makeatother

\graphicspath{{eps/}}

\renewcommand{\contentsname}{目錄}
\renewcommand{\listfigurename}{圖目錄}
\newcommand{\LOFname}{圖目錄}
\newcommand{\LOTname}{表目錄}
\renewcommand{\listtablename}{表目錄}
\renewcommand{\chaptername}{章}
\renewcommand{\appendixname}{附錄}
\newcommand{\appendices}{附錄}
\newcommand{\pagename}{頁}
\renewcommand{\glossaryname}{詞彙表}

\begin{document}

\title{Python 計算實務}
\author{Yung-Yu Chen}
\date{2016.8.1}

\maketitle

\tableofcontents
%\listoffigures

\hspace{.5cm}

\frontmatter

\chapter*{前言}
\addcontentsline{toc}{chapter}{前言}

用 Python 作應用計算。

\mainmatter

\part{執行環境}

\chapter{Anaconda}

安裝 Anaconda 與命令列執行模式。

\chapter{IPython 互動模式}

\chapter{Jupyter Notebook}

\part{資料}

\chapter{數值}

\chapter{字串}

字串與位元組 (bytes)。

\chapter{數組}

\chapter{列表}

\chapter{字典}

\part{程式結構}

\chapter{流程控制}

\chapter{函式}

\chapter{模組}

匯入。

\chapter{類別}

初始化。

\chapter{套件}

相對匯入。

避免全域變數。

\chapter{NumPy 陣列}

\part{數值計算}

\chapter{陣列檔案處理}

\chapter{字串剖析載入}

\chapter{折線圖}

\chapter{長條圖}

\part{網路通訊}

\chapter{載入網路資料}

JSON API。

\backmatter

\end{document}

% vim: set nobomb ff=unix fenc=utf8:
