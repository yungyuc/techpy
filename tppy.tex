%%
%% Copyright (c) 2016 Yung-Yu Chen <yyc@solvcon.net>
%%
%% All rights reserved.
%%

\documentclass[a4paper,12pt]{book}
%\documentclass[a4paper,12pt]{article}
%\documentclass[a4paper,12pt,dvips]{article}

\usepackage[textwidth=6.5in,textheight=9in]{geometry}

\usepackage{fontspec}
\usepackage{xeCJK}
\setCJKmainfont{Noto Sans CJK TC}
%\setCJKmainfont{Microsoft JhengHei}
%\setCJKmainfont{LiHei Pro}
%\setromanfont{LiHei Pro} % 儷黑Pro
%\setmonofont{Courier New} % 等寬字型
\XeTeXlinebreaklocale "zh"
\XeTeXlinebreakskip = 0pt plus 1pt

\usepackage[nopostdot]{glossaries}

\usepackage[colorlinks=true]{hyperref}
\usepackage{amsmath}
\usepackage{amssymb}
\usepackage{amsthm}
\usepackage[monochrome]{color}

\usepackage{graphicx}     % From LaTeX distribution
\graphicspath{{eps/}}
\usepackage[center,footnotesize]{caption}
\usepackage[subrefformat=parens]{subcaption}
%\usepackage{subfigure}    % From CTAN/macros/latex/contrib/supported/subfigure
\usepackage{pst-all}      % From PSTricks
\usepackage{pst-poly}     % From pstricks/contrib/pst-poly
\usepackage{multido}      % From PSTricks

\usepackage{xCJKnumb}
\usepackage{titlesec}
\usepackage{titletoc}
%\titleformat{\part}{\centering\Huge\bfseries}{第\,\thepart\,部分}{1em} {}
%\titleformat{\chapter}{\centering\huge\bfseries}{第\,\thechapter\,章}{1em} {}
%\titleformat{\chapter}{\centering\Huge\bfseries}{第\,\xCJKnumber{\thechapter}\,章}{1em} {}

\titleclass{\part}{top}
\titleformat{\part}%[display]
  {\normalfont\huge\bfseries}
  {第\xCJKnumber{\arabic{part}}部分}
  {1em} {}
\titlespacing*{\part}{0pt}{50pt}{40pt}

\titleclass{\chapter}{straight}
\titleformat{\chapter}%[display]
  {\normalfont\huge\bfseries}
  {第\xCJKnumber{\arabic{chapter}}章}
  {1em} {}
\titlespacing*{\chapter} {0pt}{50pt}{40pt}

\usepackage{parskip}

\usepackage{etoolbox}
\makeatletter
%\patchcmd{\chapter}{}{}{}{}
\patchcmd{\chapter}{\if@openright\cleardoublepage\else\clearpage\fi}{}{}{}
\makeatother

\renewcommand{\contentsname}{目錄}
\renewcommand{\listfigurename}{圖目錄}
\newcommand{\LOFname}{圖目錄}
\newcommand{\LOTname}{表目錄}
\renewcommand{\listtablename}{表目錄}
\renewcommand{\chaptername}{章}
\renewcommand{\appendixname}{附錄}
\newcommand{\appendices}{附錄}
\newcommand{\pagename}{頁}
\renewcommand{\glossaryname}{詞彙表}

\begin{document}

\title{Python 計算實務}
\author{Yung-Yu Chen}
\date{2016.8.1}

\maketitle

\tableofcontents
%\listoffigures

\hspace{.5cm}

\frontmatter

\chapter*{前言}
\addcontentsline{toc}{chapter}{前言}

本書是為了沒有學習過任何程式語言,尤其是非資訊科系的大學部同學,
所提供的 Python 入門教材。

Python 是一個簡潔、自然的語言,得到程式員長期的喜愛。
近十年來,Python 都是 TIOBE 社群指標\cite{tiobe_index}中的前十名,
並兩度入選年度程式語言 ("Programming Language of the Year")。
IEEE 最近公布了一項調查\cite{ieee_pl_2016},
其中 Python 是第三受歡迎的程式語言,
僅次於 C 和 Java,並且是最受歡迎的動態型別程式語言
(簡稱動態語言,dynamic language)。

在動態語言中,Python 簡潔的特性讓它特別易於學習。
人類使用程式語言,是為了驅使電腦執行作業。
但實際在開發軟體時,花在閱讀程式碼的時間,比寫出程式碼的時間多了很多。
Python 語言在設計時就考慮了這一點,使它容易被閱讀,進而也容易被撰寫。
用 C 或是 C++ 要花一個小時寫的程式,使用 Python,常常只需要五分鐘。

Python 的簡潔也讓用它寫的程式容易被維護。
許多人不了解軟體開發,以為程式寫出來以後就不需要改了。
這完全錯誤。即使開發完成的系統,已經不需要增加新的功能,
還是必須時常修補。
如果程式寫得不容易讀懂,時間一久,連原本的開發者也會難以修改。

因為 Python 易學、易維護,許多複雜的計算系統都選擇使用 Python,
為進階使用者提供應用程式介面 (API)。
這些計算系統,為了執行效率,往往選擇使用 C++ 開發。
C++ 的執行效率雖然極好,但是非常複雜難學,很難用來提供高階的 API。
所以許多這些系統會提供 Python 介面,讓使用者寫程式操作。
因為這樣的應用很常見,甚至出現了一套名為 Boost.Python 的程式庫,
專門協助 C++ 程式建立 Python 介面。

\mainmatter

\part{資料}

\chapter{執行環境}

說明 Anaconda 的安裝方式,以及命令列與 Jupyter Notebook 的執行。

\chapter{字串}

Python 使用萬國碼 (Unicode) 字串 (string)。
但萬國碼字串並不是最基礎的字串表示方式。
為了與低階的程式相容 (例如 C),Python 另外提供一種位元組 (bytes) 容器。

\chapter{整數與布林運算}

四則運算。

布林代數與位元運算。

\chapter{實數運算}

四則運算。包含整數與實數的混合運算。

大整數。

有理數。

\chapter{序列}

\section{數組}

\section{列表}

\chapter{字典}

\chapter{集合}

\part{程式結構}

\chapter{流程控制}

縮排。

\section{條件判斷}

\section{迴圈}

\chapter{函式}

\section{遞迴}

\chapter{模組}

匯入。

\chapter{套件}

相對匯入。

避免全域變數。

\chapter{類別}

初始化。

\part{數值計算}

\chapter{NumPy 陣列}

\chapter{陣列檔案處理}

\chapter{折線圖}

\chapter{長條圖}

\part{網路通訊}

\chapter{載入網路資料}

JSON API。

\backmatter

\addcontentsline{toc}{chapter}{參考資料}
\bibliographystyle{unsrt}
\bibliography{bibliography}

\end{document}

% vim: set nobomb ff=unix fenc=utf8:
