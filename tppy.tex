%%
%% Copyright (c) 2016 Yung-Yu Chen <yyc@solvcon.net>
%%
%% All rights reserved.
%%

\documentclass[a4paper,12pt]{book}
%\documentclass[a4paper,12pt]{article}
%\documentclass[a4paper,12pt,dvips]{article}

\usepackage[textwidth=6.5in,textheight=9in]{geometry}
\usepackage[colorlinks=true]{hyperref}
\usepackage[monochrome]{color}

\usepackage{fontspec}
\usepackage{xeCJK}

%\setmainfont{Calibri}
\setmainfont{Times New Roman}
\setCJKmainfont{Songti TC}
%\setCJKmainfont{Noto Sans CJK TC}
%\setCJKmainfont{Microsoft JhengHei}
%\setCJKmainfont{LiHei Pro}
%\setromanfont{LiHei Pro} % 儷黑Pro
%\setmonofont{Courier New} % 等寬字型
\XeTeXlinebreaklocale "zh"
\XeTeXlinebreakskip = 0pt plus 1pt

\usepackage[nopostdot]{glossaries}

\usepackage{amsmath}
\usepackage{amssymb}

\usepackage{amsthm}
\theoremstyle{definition}
\newtheorem{example}{範例}[chapter]

\usepackage{graphicx}     % From LaTeX distribution
\graphicspath{{eps/}}

\usepackage[center,footnotesize]{caption}
\usepackage[subrefformat=parens]{subcaption}
%\usepackage{subfigure}    % From CTAN/macros/latex/contrib/supported/subfigure
\usepackage{pst-all}      % From PSTricks
\usepackage{pst-poly}     % From pstricks/contrib/pst-poly
\usepackage{multido}      % From PSTricks

\usepackage{xCJKnumb}
\usepackage{titlesec}
\usepackage{titletoc}
%\titleformat{\part}{\centering\Huge\bfseries}{第\,\thepart\,部分}{1em} {}
%\titleformat{\chapter}{\centering\huge\bfseries}{第\,\thechapter\,章}{1em} {}
%\titleformat{\chapter}{\centering\Huge\bfseries}{第\,\xCJKnumber{\thechapter}\,章}{1em} {}

\titleclass{\part}{top}
\titleformat{\part}%[display]
  {\normalfont\huge\bfseries}
  {第\xCJKnumber{\arabic{part}}部分}
  {1em} {}
\titlespacing*{\part}{0pt}{50pt}{40pt}

\titleclass{\chapter}{straight}
\titleformat{\chapter}%[display]
  {\normalfont\huge\bfseries}
  {第\xCJKnumber{\arabic{chapter}}章}
  {1em} {}
\titlespacing*{\chapter} {0pt}{50pt}{40pt}

\usepackage{parskip}

\usepackage{etoolbox}
\makeatletter
%\patchcmd{\chapter}{}{}{}{}
\patchcmd{\chapter}{\if@openright\cleardoublepage\else\clearpage\fi}{}{}{}
\makeatother

\renewcommand{\contentsname}{目錄}
\renewcommand{\listfigurename}{圖目錄}
\newcommand{\LOFname}{圖目錄}
\newcommand{\LOTname}{表目錄}
\renewcommand{\listtablename}{表目錄}
\renewcommand{\chaptername}{章}
\renewcommand{\appendixname}{附錄}
\newcommand{\appendices}{附錄}
\newcommand{\pagename}{頁}
\renewcommand{\glossaryname}{詞彙表}

\makeatletter
\renewcommand\p@chapter{\xCJKnumber{\arabic{chapter}}\expandafter\@gobble}
\makeatother

\usepackage{cleveref}

\begin{document}

\title{Python 計算實務}
\author{Yung-Yu Chen}
\date{2017.2.20}

\maketitle

\tableofcontents
%\listoffigures

\hspace{.5cm}

\frontmatter

\chapter{前言}
%\addcontentsline{toc}{chapter}{前言}

這是一本學習 Python 程式語言的入門書籍。
本書儘可能涵蓋 Python 所有的應用領域,期望帶給讀者完整的介紹。

%
% Python 簡介
%

Python 程式語言發展至今已逾 25 年。
1989 年 12 月,
Guido van Rossum 在荷蘭的 Centrum Wiskunde \& Informatica (CWI) 開始了
Python 的開發工作,並在 1990 年發行 CWI 內部版本。
它的第一個公開版本是 0.9.0,
發行於 1991 年 2 月 20 日\cite{rossum_brief_2009}。
相較於 1996 年發行 1.0 版的 Java,Python 的歷史還更久一點。

Python 簡潔、自然、功能繁多,是理想的入門教學語言。
Python 是動態型別程式語言
(dynamic-typing programming language;簡稱動態語言;dynamic language)。
當我們使用動態語言撰寫程式的時候,
首先會注意到的就是它不必宣告 (declare)、直接使用變數 (variable) 的特性。
在電腦的世界裡,變數是用來識別資料的方式。
每項資料都被賦予型態 (type),宣告即是在撰寫程式的當下指定變數型態的方法。
電腦必須知道變數的型態,才能導向至正確的邏輯,處理資料。
在動態語言中,變數的型態會在程式執行的時候自動被決定。
在那之前 (例如撰寫程式的當下),變數的型態並未確定,也就不需要宣告了。

每一種程式語言都需要編譯器 (compiler),
把人類能讀懂的原始碼 (source code)
編譯 (compile) 成電腦能執行的機器碼 (machine code)。
但動態語言一般採取另一種稱作直譯器 (interpreter) 的實作方法。
直譯器在讀取原始碼的同時,直接把原始碼轉換成機器碼,讓電腦執行。
編譯與直譯這兩種模式最大的差別,是編譯器能一次掃瞄整份原始碼,
全面性地最佳化 (optimize) 之後再生成機器碼。
直譯器可見的原始碼有限,只能作局部的最佳化,因此執行效率很差。

因為編譯器協助動態語言的部分有限,傳統上它不是學校課程所選擇的程式語言。
因此,Python 過去不像 C、C++、Java 等語言那麼出名。
近年來,由於編程的應用與普及,Python 開始在學校為人所知。

本書透過演繹和範例展示 Python 的功能,
使讀著無論採取課堂講授或自主學習的方式,
都能確實地習得基本 Python 編程的能力。

目前 Python 發展的重心是 Python 3。Python 3 發行於 2008 年。
為了擺脫過去的設計缺陷,Python 3 具有不相容於 Python 2 的特性。
常用的 Python 程式庫都已經移植到了 Python 3,
是故一般的使用者可以放心使用 Python 3 的好處。
本書也針對 Python 3 撰寫。

%
% 章節編排
%

讀者不需要程式語言的背景,也能使用本書學會 Python。
從一開始的第\ref{c:runtime}、\ref{c:editor}章,帶領讀者建立執行與開發環境。
第\ref{c:arithmic}、\ref{c:container}、\ref{c:control}、\ref{c:structure}、%
\ref{c:exception}、\ref{c:inout}章%
實際說明如何使用 Python 執行計算。
第\ref{c:deployment}、\ref{c:debug}章介紹必備的軟體工程技能。
最後在第\ref{c:regexp}、\ref{c:meta}、\ref{c:concurrency}、%
\ref{c:extension}、\ref{c:beyond}章,介紹進階的 Python 功能。

%
% 學習目的
%

電腦 (computer) 豐富人類的生活,
使我們無法拒絕在任何一個可能使用電腦協助我們的場合,利用電腦幫助我們。

計算思維 (computational thinking)。

本書是為了沒有學習過任何程式語言,
尤其是非資訊科系的大學部同學或中學生所提供的 Python 入門教材。

Python 是一個簡潔自然的語言,長期得到程式員的喜愛。
近十年來,Python 都是
TIOBE 社群指標\footnote{\url{http://www.tiobe.com/tiobe-index/}}中的前十名,
並兩度入選年度程式語言 (``Programming Language of the Year'')。
Institute of Electrical and Electronics Engineers (IEEE) 發現
Python 受程式員歡迎的程度僅次於 C 和 Java,
是最受歡迎的動態語言 (dynamic language) \cite{cass_interactive:_2016}。

程式語言為了驅使電腦而存在。
但實際在開發軟體時,閱讀程式碼的時間遠多於寫出程式碼的時間。
考慮到這一點,Python 被設計成容易閱讀的程式語言。
用 C 或是 C++ 要花一個小時寫的程式,使用 Python,常常只需要五分鐘。

Python 的可讀性使 Python 程式容易被維護。
程式寫出來以後仍然需要時常維護修改。
維護時不會增加新功能,但要閱讀程式,並修改那些一開始寫得不夠完美的部分。
沒有人能一次就把程式寫到盡善盡美。如同作家無時無刻都在修改文稿一樣。
如果程式寫得不容易讀懂,時間一久,連原本的開發者也會難以修改。

因為 Python 易學、易維護,許多複雜的計算系統都選擇使用 Python,
為進階使用者提供應用程式介面 (API)。
這些計算系統,為了執行效率,往往選擇使用 C++ 開發。
C++ 的執行效率雖然極好,但是非常複雜難學,很難用來提供高階的 API。
所以許多這些系統會提供 Python 介面,讓使用者寫程式操作。
因為這樣的應用很常見,也有許多程式庫專門協助 C++ 程式建立 Python 介面。

\mainmatter

\part{認識環境}
%
\label{p:environment}

\chapter{執行環境}
%
\label{c:runtime}

本書的範例程式是在 Ubuntu Linux 16.04 LTS 或 Mac OSX 10.12 下開發測試。
讀者最好在相同的環境下執行。

\section{Ubuntu}

\section{VirtualBox}

\section{Anaconda}

\begin{example}
使用命令列輸出 Hello World
\end{example}

\section{命令列環境}

\chapter{文字編輯器}
%
\label{c:editor}

\section{VIM}

\section{Jupyter Notebook}

\part{核心功能}
%
\label{p:core}

\chapter{數值和字串、表達式和陳述式}
%
\label{c:arithmic}

\section{整數與布林運算}

布林代數與位元運算。

\begin{example}
位元平移
\end{example}

\begin{example}
實作真值表
\end{example}

\section{實數運算}

四則運算。包含整數與實數的混合運算。

\begin{example}
實作雙元運算子計算機
\end{example}

\begin{example}
真實除法與整數除法不一樣
\end{example}

\section{字串}

Python 使用萬國碼 (Unicode) 字串 (string)。
但萬國碼字串並不是最基礎的字串表示方式。
為了與低階的程式相容 (例如 C),Python 另外提供一種位元組 (bytes) 容器。

\begin{example}
計算字串的長度
\end{example}

\begin{example}
計算字元出現的次數
\end{example}

\chapter{容器、資料結構、演算法}
%
\label{c:container}

\section{序列}

\section{數組}

\section{列表}

\section{字典}

\section{集合}

\chapter{迴圈與條件判斷}
%
\label{c:control}

縮排。

\chapter{函式、類別、物件}
%
\label{c:structure}

\section{遞迴}

\chapter{例外處理}
%
\label{c:exception}

迭代器與產生器

\chapter{檔案與輸出入}
%
\label{c:inout}

\part{軟體工程}
%
\label{p:software}

\chapter{版本控制、佈署、模組、套件}
%
\label{c:deployment}

匯入。

相對匯入。

避免全域變數。

\chapter{測試與除錯}
%
\label{c:debug}

\part{進階功能}
%
\label{p:advance}

\chapter{常規表示式}
%
\label{c:regexp}

\chapter{詮釋編程}
%
\label{c:meta}

\chapter{並發計算}
%
\label{c:concurrency}

\chapter{延伸模組}
%
\label{c:extension}

\chapter{綜合應用}
%
\label{c:beyond}

\begin{itemize}

\item 命令列介面設計。

\item 文字資料處理。載入網路資料;JSON。

\item NumPy 多維連續陣列。

\item 線性系統。

\end{itemize}

\backmatter

\addcontentsline{toc}{chapter}{參考資料}
\bibliographystyle{unsrt}
\bibliography{bibliography}

\end{document}

% vim: set nobomb ff=unix fenc=utf8:
