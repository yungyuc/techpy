%%
%% Copyright (c) 2016 Yung-Yu Chen <yyc@solvcon.net>
%%
%% All rights reserved.
%%

\documentclass[a4paper,12pt]{book}
%\documentclass[a4paper,12pt]{article}
%\documentclass[a4paper,12pt,dvips]{article}

\usepackage[textwidth=6.5in,textheight=9in]{geometry}
\usepackage[colorlinks=true]{hyperref}
\usepackage[monochrome]{color}

\usepackage{fontspec}
\usepackage{xeCJK}

%\setmainfont{Calibri}
\setmainfont{Times New Roman}
\setCJKmainfont{Songti TC}
%\setCJKmainfont{Noto Sans CJK TC}
%\setCJKmainfont{Microsoft JhengHei}
%\setCJKmainfont{LiHei Pro}
%\setromanfont{LiHei Pro} % 儷黑Pro
%\setmonofont{Courier New} % 等寬字型
\XeTeXlinebreaklocale "zh"
\XeTeXlinebreakskip = 0pt plus 1pt

\usepackage[nopostdot]{glossaries}

\usepackage{amsmath}
\usepackage{amssymb}

\usepackage{amsthm}
\theoremstyle{definition}
\newtheorem{example}{範例}[chapter]

\usepackage{graphicx}     % From LaTeX distribution
\graphicspath{{eps/}}

\usepackage[center,footnotesize]{caption}
\usepackage[subrefformat=parens]{subcaption}
%\usepackage{subfigure}    % From CTAN/macros/latex/contrib/supported/subfigure
\usepackage{pst-all}      % From PSTricks
\usepackage{pst-poly}     % From pstricks/contrib/pst-poly
\usepackage{multido}      % From PSTricks

\usepackage{minted}

\usepackage{xCJKnumb}
\usepackage{titlesec}
\usepackage{titletoc}
%\titleformat{\part}{\centering\Huge\bfseries}{第\,\thepart\,部分}{1em} {}
%\titleformat{\chapter}{\centering\huge\bfseries}{第\,\thechapter\,章}{1em} {}
%\titleformat{\chapter}{\centering\Huge\bfseries}{第\,\xCJKnumber{\thechapter}\,章}{1em} {}

\titleclass{\part}{top}
\titleformat{\part}%[display]
  {\normalfont\huge\bfseries}
  {第\xCJKnumber{\arabic{part}}部分}
  {1em} {}
\titlespacing*{\part}{0pt}{50pt}{40pt}

\titleclass{\chapter}{straight}
\titleformat{\chapter}%[display]
  {\normalfont\huge\bfseries}
  {第\xCJKnumber{\arabic{chapter}}章}
  {1em} {}
\titlespacing*{\chapter} {0pt}{50pt}{40pt}

\usepackage{parskip}

\usepackage{etoolbox}
\makeatletter
%\patchcmd{\chapter}{}{}{}{}
\patchcmd{\chapter}{\if@openright\cleardoublepage\else\clearpage\fi}{}{}{}
\makeatother

\renewcommand{\contentsname}{目錄}
\renewcommand{\listfigurename}{圖目錄}
\newcommand{\LOFname}{圖目錄}
\newcommand{\LOTname}{表目錄}
\renewcommand{\listtablename}{表目錄}
\renewcommand{\chaptername}{章}
\renewcommand{\appendixname}{附錄}
\newcommand{\appendices}{附錄}
\newcommand{\pagename}{頁}
\renewcommand{\glossaryname}{詞彙表}

\makeatletter
\renewcommand\p@chapter{\xCJKnumber{\arabic{chapter}}\expandafter\@gobble}
\makeatother

\usepackage{cleveref}

\newenvironment{notice}
{
\begin{center}
\begin{minipage}[t]{0.9\textwidth}
\noindent 注意
\vskip 1em

}
{
\end{minipage}
\end{center}
}

\begin{document}

\title{Python 計算實務}
\author{Yung-Yu Chen}
\date{2017.2.28}

\maketitle

\tableofcontents
%\listoffigures

\hspace{.5cm}

\frontmatter

\chapter{前言}
%\addcontentsline{toc}{chapter}{前言}

這是一部為了沒有學習過任何程式語言的讀者、
非資訊科系的大學部同學或中學生所準備的 Python 入門教材。

撰寫電腦程式指揮電腦,完成大量的資訊處理工作,是當世必備的技能。
電腦已經成為容易取得的設備,但各人指揮電腦的功夫的差異很大。
這造成各行各業中,哪怕只是多一點點操作電腦的技術,都成為巨大的生產力優勢。
寫程式是最有彈性的電腦使用方式。
學會寫程式,就是用最高效率的方法,掌握了最方便好用的工具。

程式編寫的本質是資訊處理的自動化。
在數位電腦 (digitial computer) 發明之前,人類依靠肉身處理資訊。
好比印刷術發明之前,若不依靠抄寫員複製書籍,那就只好以背誦或傳唱傳遞知識。
資訊處理工作要求大量人腦所不擅長的重複作業,
而不斷精進的半導體電路,恰恰最擅長重複。
我們在學習編寫程式時,
也必然得學會基本的電腦科學和一點數學。

寫程式這種技能不但要求高度的知性 (intellect),更是一項強調熟習的技藝 (craft)。
Dijkstra 為了寫下一個簡單迴圈的數學證明,
耗費了一整頁的篇幅\cite{dahl_structured_1972}。
實際上在編寫程式的時候,哪裡可能花這麼多力氣處理單單一個迴圈呢。
學習程式語言,許多的時間應該花在練習正確的程式成語 (idiom) 與其脈絡。
藉由不斷閱讀好程式的寫法,程式員才能精進編程的能力。

從入門的時候就開始程式寫作的鍛鍊,養成好習慣,
不但增加學習的效率,也利於有志於專業編程者的長遠發展。
Python 非常適於向上向下的擴展。
在 Python 上面建立了良好的基礎之後,
可以迅速抓住注重細節的 C 與 C++ 的技術重點。
使用 Python 建造系統的時候,持續磨練技藝,
不但強化系統,也增進實作能力甚或語言本身。

為了有效地學習基本的 Python 編程,
希望讀者在閱讀本書的時候,同時打開電腦,實際輸入程式練習。
本書會交錯摻雜各種例題和講解,以便讀者練習。

本書針對 Python 3 撰寫。
初學的讀者要注意,Python 3 與舊版的 Python 2 有細微的不同。
Python 發展至今已歷 25 年,更新無數。
為了維持舊有程式的執行平台,Python 2 仍在維護中。
但自 2008 以後,所有的新功能都加在 Python 3,而 Python 2 的版本也停在了 2.7。
Python 最新的版本是 3.6。本書的所有範例以其為準。

Python 學起來簡單,寫起來也簡單。
近十年來,Python 都是 TIOBE 社群指標中的前十名,
並兩度入選年度程式語言
(``Programming Language of the Year'')\footnote{\url{http://www.tiobe.com/tiobe-index/}}。
Institute of Electrical and Electronics Engineers (IEEE) 發現
Python 受程式員歡迎的程度僅次於 C 和 Java,
是最受歡迎的動態語言 (dynamic language) \cite{cass_interactive:_2016}。

Python 易學易寫的秘密在於它的易讀性 (readability)。
易讀的特性也讓它成為初學程式設計的最佳語言。
無論新手熟工,總是讀懂了問題才能設法解決。
如果連書本裡面的範例程式都讀不懂,如何學習呢。
這看似淺明的道理是程式設計的精髓。
Python 的設計者把易讀性放在語言的核心。
語言的使用者當掌握要義。

為底層的高速 C/C++ 程式庫 (library) 提供應用程式介面 
(Application Programming Interface, API),也是 Python 的主要應用之一%
\footnote{\url{https://www.tensorflow.org/} 為一例。}。
這些計算系統,往往為了執行效率而選擇使用 C++ 程式語言開發。
C++ 的執行效率極好但複雜難學,無法提供簡明的高階 API。
Python 本身是以 C 語言開發,隨附了完善的 C API,
允許以 C 語言延伸 (extend) Python,
或在以 C 語言撰寫的系統裡嵌附 (embed) Python。
對 C++ 而言,除了同樣的 C API 之外,
也能使用其它程式庫進一步簡化延伸與嵌附的工作。

雖然 Python 已是一個通用程式語言 (generic language),
功能極完整,可以執行幾乎任何任務,
但以 C/C++ 延伸 Python 的能力,仍是資深程式員不可或缺的。
許多資深的 Python 程式員熟悉 Python 的底層。
讀到一段 Python 程式時,他們可以概略轉譯出底層的 C 語言實作,
快速分析程式的效能。
這種與 C/C++ 相互合作的結構,是 Python 的一大特色。
在高效能計算 (high-performance computing, HPC) 領域內,
有「以 Python 寫 C」之說,實不為假。
這不是指以 Python 產生 C 程式碼再編譯執行,
而是撰寫原生的 Python 程式碼,
讓 Python 直譯器最後組合出來的底層呼叫,擁有近於原生 C 程式的速度。

Python 易讀、易寫、易擴充,是最佳入門程式語言,也是專業程式員的最佳工具。
如此方便的語言工具,更能協助我們應用計算思維 (computational thinking),
處理過去單憑人類的肉體無法解決的問題。
從彈道計算、文書處理、智慧型手機,到自動駕駛汽車,
電腦的應用大大地增幅了人類的能力。
我們沒有理由認為電腦的應用會止於我們目前所見。
我們要為任何一種領域思考,
是否能從計算 (computation) 的角度重塑工作模式,
完成過去所想像不到的事情。

希望 Python 成為讀者最有力的計算工具。

\mainmatter

\chapter{執行環境}
%
\label{c:runtime}

Python 簡潔、自然、功能繁多,是理想的程式設計教學語言。
分類上,Python 屬於動態型別程式語言
(dynamic-typing programming language;簡稱動態語言;dynamic language),
原始碼可以直接被執行。
一般來說,程式語言需要編譯器 (compiler),
把人類能讀懂的原始碼 (source code)
編譯 (compile) 成電腦能執行的機器碼 (machine code)。
但動態語言一般採取直譯器 (interpreter) 的實作方法。
直譯器在讀取原始碼的同時,直接把原始碼轉換成機器碼,讓電腦執行。
這兩種模式最大的差別,是編譯器能一次掃瞄整份原始碼,
全面性地最佳化 (optimize) 之後再生成機器碼。
直譯器可見的原始碼有限,只能作局部的最佳化,因此執行效率較差。
但經過本書的介紹,你會了解為什麼語言本身效率較 C/C++ 差的 Python,
通常可以實作出更為有用的程式。
而有時,Python 程式的執行速度還可以超過 C/C++。

Python 程式語言發展至今逾 25 年。
1989 年 12 月,
Guido van Rossum 在荷蘭的 Centrum Wiskunde \& Informatica (CWI) 開始了
Python 的開發工作,並在 1990 年發行 CWI 內部版本。
它的第一個公開版本是 0.9.0,
發行於 1991 年 2 月 20 日\cite{rossum_brief_2009}。
相較於 1996 年發行 1.0 版的 Java,Python 的歷史還更久一點。
在這一段長時間的發展中,
Python 累積了大量的程式庫 (library) 與廣大的使用者社群,
成為世界上應用最多、最受歡迎的程式語言之一。

本書的範例程式均在 Ubuntu Linux 16.04 LTS (Xenial Xerus)\footnote{\url{https://www.ubuntu.com/}.}
下開發測試。
讀者應在相同的環境下執行。
為了方便讀者學習,本章將講述如何在 Windows、Linux、macOS 等各種系統下,
設定可用於本書的執行環境。

\section{使用 Anaconda}
%
\label{c:runtime:anaconda}

由於 Python 是一套開放原始碼 (開源;open source) 的軟體,
它在同樣是開源的 Ubuntu Linux 作業系統中擁有很好的支援。
本書選擇了目前最新的 Long Term Support (LTS) Ubuntu 版本,
16.04,作為執行平台。
在相同的作業系統之下執行本書中的範例程式,
初學 Python 的讀者較不容易遇見軟體版本衝突而引發的程式錯誤。

採用 Ubuntu Linux 的另一個原因,
是它有較好的命令行介面 (command-line interface, CLI) 支援。
顧名思義,命令行介面是由使用者輸入一個指令,而電腦回應一個動作的介面。
雖然命令行介面在互動性上大大的不如圖形化介面 (graphical user interface, GUI),
它的操作過程卻很容易複製。
使用命令行介面,讀者可以輕易地依照本書的說明,對電腦下達相同的指令。

為了使用最新的 Python,我們要安裝
Anaconda Distribution (\url{https://www.continuum.io/anaconda-overview})。
Anaconda 是 Continuum Analytics\footnote{\url{https://www.continuum.io}.}
公司所提供免費開源的跨平台 Python 執行環境。
Anaconda 內含了許多 Python 及與 Python 有關的軟體套件。
透過 Anaconda,除了 Linux 之外,也能在 Windows 和 macOS 上安裝相同的軟體套件。

除了版本和跨平台以外,Anaconda 還有另外兩大優點:
%
\begin{enumerate}
%
\item 可以直接安裝在使用者目錄內,不需要 root 權限。
root 是 Linux 具有最高權限的系統管理帳號。
%
\item 在單一的 Anaconda 安裝內,可以建立無數個獨立的執行環境。
在各個獨立的執行環境內,可以安裝不同的軟體套件。
在開發或測試程式時,這是很方便的功能。
%
\end{enumerate}

依據 \url{https://www.continuum.io/downloads} 的說明,
即可下載 Anaconda 並安裝。
也可以打開 Ubuntu 的終端機 (terminal;提供命令行介面),
輸入以下的指令安裝 (此處假設你使用的是一般 x86\_64 架構的 PC):

\begin{minted}{bash}
$ mkdir -p ~/tmp
$ curl -SL -o ~/tmp/anaconda.sh \
https://repo.continuum.io/archive/\
Anaconda3-4.3.0-Linux-x86_64.sh
$ bash ~/tmp/anaconda.sh -b -p $HOME/anaconda
\end{minted}
%stopzone

%\begin{center}
%\begin{minipage}[t]{0.8\textwidth}
%\noindent 注意 \newline
%
%\noindent
\begin{notice}
\noindent 值此寫作之時,Anaconda 最新的版本是 4.3。但它版本推進的速度很快。
若最新的版本已經不是 4.3,請安裝更新版。
\end{notice}
%\end{minipage}
%\end{center}

以上的指令表示我們要把 Anaconda 安裝到 \verb+~/anaconda+ 目錄下。
在使用所安裝的 Python 之前,需要調整執行檔搜尋路徑。
請在 \verb+~/.bashrc+ 的檔尾加上一行:

\begin{minted}{bash}
export PATH=$HOME/anaconda/bin:$PATH
\end{minted}
%stopzone

檔案改好了以後,關閉再重新開啟終端機,就可以使用安裝好的 Python 了:

\begin{minted}{bash}
$ which python3
/home/ubuntu/anaconda/bin/python3
$ python3
Python 3.6.0 |Anaconda 4.3.0 (64-bit)| (default, Dec 23 2016, 12:22:00) 
[GCC 4.4.7 20120313 (Red Hat 4.4.7-1)] on linux
Type "help", "copyright", "credits" or "license" for more information.
>>> 
\end{minted}
%stopzone

還記得 Python 有第二與第三版的分別嗎?
為了避免執行到錯誤的版本,
本書都使用「\verb+python3+」來明確指明使用第三版的 Python。

\section{VirtualBox 虛擬機器}
%
\label{c:runtime:vbox}

\section{使用既有套件}

Anaconda 與 PIP。

\begin{example}
使用命令列輸出 Hello World
\end{example}

\section{命令列互動模式}

\section{習題}

\chapter{撰寫程式}
%
\label{c:write}

當我們使用動態語言撰寫程式的時候,
首先會注意到的就是它不必宣告 (declare)、直接使用變數 (variable) 的特性。
在電腦的世界裡,變數是用來識別資料的方式。
每項資料都被賦予型態 (type),宣告即是在撰寫程式的當下指定變數型態的方法。
電腦必須知道變數的型態,才能導向至正確的邏輯,處理資料。
在動態語言中,變數的型態會在程式執行的時候自動被決定。
在那之前 (例如撰寫程式的當下),變數的型態並未確定,也就不需要宣告了。

\section{VIM}

\section{Jupyter Notebook}

\section{單步追蹤}

\chapter{表示式與陳述式}
%
\label{c:arithmic}

\section{整數與布林運算}

布林代數與位元運算。

\begin{example}
位元平移
\end{example}

\begin{example}
實作真值表
\end{example}

\section{實數運算}

四則運算。包含整數與實數的混合運算。

\begin{example}
實作雙元運算子計算機
\end{example}

\begin{example}
真實除法與整數除法不一樣
\end{example}

\section{字串}

Python 使用萬國碼 (Unicode) 字串 (string)。
但萬國碼字串並不是最基礎的字串表示方式。
為了與低階的程式相容 (例如 C),Python 另外提供一種位元組 (bytes) 容器。

\begin{example}
計算字串的長度
\end{example}

\begin{example}
計算字元出現的次數
\end{example}

\chapter{流程控制}
%
\label{c:flow}

縮排。

\chapter{迴圈}
%
\label{c:loop}

\chapter{函式}
%
\label{c:function}

\section{遞迴}

\chapter{列表}
%
\label{c:list}

\chapter{數組}
%
\label{c:tuple}

\chapter{字典}
%
\label{c:dict}

\chapter{集合}
%
\label{c:set}

\chapter{類別}
%
\label{c:class}

\section{方法}

\chapter{例外處理}
%
\label{c:exception}

\section{迭代器}

\section{產生器}

\chapter{檔案輸出入}
%
\label{c:file}

介紹檔案物件。

例題:模擬檔案物件。

\chapter{存取網頁資料}
%
\label{c:web}

\section{JSON 資料格式}

\chapter{版本控制}
%
\label{c:version}

\chapter{模組與套件}
%
\label{c:module}

匯入與相對匯入。

\chapter{單元測試}
%
\label{c:unittest}

\chapter{陣列處理}
%
\label{c:array}

求解線性系統。

\chapter{常規表示式}
%
\label{c:regexp}

\chapter{詮釋編程}
%
\label{c:meta}

\chapter{命令列介面}
%
\label{c:cmdline}

\section{測試}

\chapter{並發計算}
%
\label{c:concurrency}

\chapter{延伸模組}
%
\label{c:extension}

\chapter{系統嵌入}
%
\label{c:embed}

\backmatter

\addcontentsline{toc}{chapter}{參考資料}
\bibliographystyle{unsrt}
\bibliography{bibliography}

\end{document}

% vim: set nobomb ff=unix fenc=utf8:
